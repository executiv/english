\documentclass{article}

\usepackage{outlines, color}

\title{\LARGE \scshape{2014 Curriculum Notes} \\ \Huge \scshape{Year 10 English}}
\author{\Large \scshape{Tyler Fernando} \\ \small \scshape{Yarra Valley Grammar}}
\date{}

\begin{document}
\maketitle

\tableofcontents

\cleardoublepage

\section{UNLOCKING ACT I}
\subsection{Curtain Up!}
\subsubsection{In one sentence, summarise the content of the Prologue to Romeo and Juliet.}
Two families are in constant conflict, and two lovers from opposite sides take their life.
\subsubsection{What do you think is the purpose of telling the whole story even before the play has begun?}
It prepares the audience for what is to come. The point of the prologue is to give people an idea about the storyline so they are able to make their own guesses of what is going to happen, essentially keeping the audience more engaged.
\subsection{What's What?}
\subsubsection{From this list of events from Act I, choose the three you think are the most important then explain your reasoning using examples where possible.}
\begin{outline}
\1 Prince Escalus declares the next to fight will die
\2 This is the reason why Old Capulet orders Tybalt to stand down, which was important in the leading up to Romeo and Juliet meeting
\1 Benvolio and Romeo plan to go to the Capulet feast that night
\2 Benvolio and Romeo, or rather Mercutio and Romeo, going to the Capulet feast was how Romeo and Juliet met, and is therefore integral to the overall story.
\1 Romeo and Juliet meet
\2 Romeo and Juliet meeting is the entire point of the story "Romeo and Juliet", and plays possibly the most important role.
\end{outline}
\subsection{Who's Who?}
\subsubsection{Describe each character with two words in each act.}
\begin{tabular}[t]{p{2.5cm} || p{2cm}*{4}{| p{2cm}}r}
Character & Act I & Act II & Act III & Act IV & Act V \\
\hline \hline
Sampson & Tyrannical, Sexist & & & & \\
\hline
Gregory & Provocative, Loyal & & & & \\
\hline
Abraham & Volatile, Quick-tempered & & & & \\
\hline
Benvolio & Apathetic, Unsympathetic & & & & \\
\hline
Tybalt & Antagonistic, Unsympathetic & & & & \\
\hline
Capulet & Commanding, Amiable & & & & \\
\hline
Lady Capulet & Vain, \ \ \ \ \ \ \ \ Self-assured & & & & \\
\hline
Montague & Wealthy, Observant & & & & \\
\hline
Lady Montague & Concerned, Objective & & & & \\
\hline
Prince Escalus & Peacekeeping, Powerful & & & & \\
\hline
Romeo & Romantic, Rebellious & & & & \\
\hline
Paris & Aristocratic, Self-absorbed & & & & \\
\hline
Nurse & Neutral, Motherly & & & & \\
\hline
Juliet & Romantic, Timid & & & & \\
\hline
Mercutio & Facetious, Unpredictable & & & & \\
\end{tabular}
\subsection{Who's Speaking?}
\subsubsection{Metaphors and similes: There are dozens of metaphors and similes in Romeo and Juliet. They serve the same function but a metaphor is more direct than a simile. A metaphor is a direct comparison between two things (for example: He was a tower of strength.). A simile is an indirect comparison between two things, indicated by the inclusion of either `like' or `as' (for example: He was like a tower of strength.).}
\begin{outline}
\1 For each scene find one metaphor and explain its meaning.
\2 ACT I, Scene I
\3
\2 ACT I, Scene II
\3
\2 ACT I, Scene III
\3
\2 ACT I, Scene IV
\3
\2 ACT I, Scene V
\3
\end{outline}
\subsubsection{Oxymorons: an oxymoron is a figure of speech in which we put together two contradictory terms. For example: Romeo utters a number of them in a row (I.i.164-170)}
\begin{outline}
\1 Why does he use them?
\2 
\1 What does each mean?
\2 
\1 What effect do they have?
\2 
\end{outline}
\subsection{The Big Picture}
\subsubsection{At the heart of the play are the two extremes of love and hatred. The play begins with an open act of hatred and ends with Juliet's observation, `My only love sprung from my only hate!'.}
\begin{outline}
\1 Look over Act I. How many references can you find to love and hate?
\2 
\1 Who speaks of love and who, of hatred?
\2 
\1 Make predictions about what might happen to those who speak of love.
\2 
\1 Make predictions about what might happen to those who speak of hatred.
\2 
\end{outline}

\newpage

\section{UNLOCKING ACT II}
\subsection{Curtain Up!}
\subsubsection{Re-read the Chorus lines at the end of Act I. It introduces Act II.}
\begin{outline}
\1 In less than fifty words, write down the meaning of this speech.
\1 How convinced are you by the assertion that Romeo `loves again'?
\end{outline}
\subsection{What's What?}
\subsubsection{By the end of Act II, Romeo and Juliet are married. Using a list of dot points, trace the events that have led to the wedding.}
\begin{outline}

\end{outline}
\subsection{Who's Who?}
The role of Friar Lawrence: We are introduced to Friar Lawrence in Act II. He has a pivotal role to play in Romeo and Juliet.
\subsubsection{Re-read his speech of II.iii.1-30. What do you learn about him from this?}
\subsubsection{Now re-read}

\section{What is a tragedy?}
\begin{outline}
\1 The main character suffers because of a personality fault and/or social conditions that make him/her unable to deal with the events that occur. 
\1 The victims of events are usually of a high standing in the community. 
\1 There may be references to the supernatural, gods, hallucinations, and insanity. 
\1 The suffering affects many innocent people, not just the main character. 
\1 The main character dies after much personal suffering (either physical or emotional).
\end{outline}

\section{Five act structure of Shakespeare's plays}
\begin{outline}
\1 ACT I
\2 Orientation -- The place and characters are introduced. Relationships between characters are established. The major themes (love, hate, betrayal) are introduced.
\1 ACT II
\2 Problems arise -- A major conflict happens (usually murder), leading to the situation heightening throughout the play until the solution at the end.
\1 ACT III
\2 Complications -- The plot becomes more complicated. Evil forces take control.
\1 ACT IV
\2 Turn around of fortunes -- Goodness begins to turn around the forces of evil. The hero moves toward his tragic end.
\1 ACT V
\2 Resolution -- The inevitable conclusion is reached as the hero dies. The characters move on to a better world, enriched by their experiences.
\end{outline}

\section{Plot Summaries}
\begin{outline}
\1 Prologue
\2 The chorus delivers these lines, briefly summarising the play
\2 Explains the families of Verona and that the love story of Romeo and Juliet will be the focus of the rest of the play.
\2 KEY QUOTES
\3 `civil blood makes civil hands unclean'
\3 `A pair of star-crossed lovers take their life'
\1 Act 1
\2 Scene 1
\3 The play begins with Sampson and Gregory (Capulets) who are out in Verona (Italy - where the play is set) amusing themselves and the audience with their wit and dirty jokes
\3 Abram (Abraham) and Balthasar (Montagues) enter the town square and they fight with Sampson and Gregory
\3 Benvolio (Montague) tries to stop the fighting
\3 Tybalt (Capulet) tries to fight Benvolio
\3 This scene shows the hatred and violence between the two families and how it involves everyone even the servants
\3 Prince enters and he is angry that this has happened again and warns them both that if they fight again, they will be executed
\3 The Montagues are worried about Romeo and ask Benvolio to help find out what is wrong with him
\2 Scene 2
\3 Lord Capulet meets Paris who has been pressuring Capulet to allow him to marry Juliet, but Capulet believes she's too young
\3 Capulet invites Paris to a feast so that he can meet Juliet
\3 The servant is left with a list of people to invite to the feast and as he cannot read asks Romeo to read it for him and the Montagues decide to go to the feast themselves.
\2 Scene 3
\3 Juliet is introduced
\3 The nurse tells a story of when Juliet was young allowing the audience to have a moment of amusement
\3 Lady Capulet informs Juliet that Paris is interested in her and Juliet promises to meet him at the feast
\2 Scene 4
\3 Romeo and Benvolio head to the feast and Mercutio who has been formally invited is with them
\3 Romeo is reluctant to go because he's too sad (about Rosaline)
\3 Romeo is worried about a dream that he had and Mercutio tells him about Queen Mab the fairy queen who brings sleepers their dreams. Every persons occupation will determine the dreas that they have, indicating that Romeo's love for Rosaline is imaginary
\3 Romeo says he has a terrible feeling and feel that this night will lead to some awful consequence
\2 Scene 5
\3 The Capulet ball
\3 Romeo sees Juliet on the dance floor and is struck by her beauty
\3 Tybalt sees Romeo and wants to fight him immediately, but Capulet stops him. He vows to get his revenge later
\3 Romeo and Juliet talk to each and Romeo kisses Juliet
\3 She is called away and he is shocked to find out that she is a Capulet
\3 Juliet questions the nurse about Romeo and is equally distressed to find out that he's a Montague
\3 KEY QUOTES
\4 `If you ever disturb our streets again, your lives shall play forfeit of the peace' -- Prince
\4 `What sadness lengthens Romeo's hours' -- Benvolio
\4 `thou canst not teach me to forget'-- Romeo (about Rosaline)
\4 `Let two more summers wither in ther pride, ere we may think her ripe to be a bride' -- Capulet (to Paris)
\4 `Thou wast the prettiest babe that e'er I nursed, an I might live to see thee married' -- Nurse (to Juliet)
\4 `The valient Paris seeks you for his love' -- Lady Capulet (to Juliet)
\4 `I have a soul of lead, so stakes me to the ground I cannot move' -- Romeo
\4 `If love be rough with you, be rough with love' -- Mercutio
\4 `True I talk of dreams; Which are the children of an idle brain begot of nothing but vain fantasy' -- Mercutio
\4 `My mind misgives. Some consequence yet hanging in the stars' -- Romeo
\4 `This is a Montague, our foe' -- Tybalt
\4 `My only love sprung from my only hate!' -- Juliet
\1 Act 2
\2 Scene 2
\3 Romeo doesn't want to leave Juliet so he jumps over the Capulet's walls and hides on the property
\3 He sees Juliet standing on her balcony and he speaks of how beautiful she is
\3 Juliet speaks about Romeo, unaware that he can hear or see her
\3 Romeo vows to be faithful to Juliet
\3 She says that she will send the nurse to him tomorrow to discuss his intentions and on which day they will marry
\2 Scene 3
\3 Romeo enters the monastery to see Friar Laurence who is attending to his herbs
\3 The Friar is amazed at how quickly Romeo had forgotten about Rosaline
\3 The Friar agrees to marry Romeo and Juliet, believing that their marriage will heal the rift between the two families
\2 Scene 5
\3 Juliet awaits the nurses return from her meeting with Romeo
\3 The nurse tells her of Romeo's plans
\2 Scene 6
\3 The Friar and Romeo wait for Juliet
\3 The Friar hopes that God approves and that this marriage will be for the best
\3 Juliet arrives and the Friar rushes them off to get married
\2 KEY QUOTES
\3 `Blind is love and best befitts the dark' -- Benvolio
\3 `What light from yonder window breaks? It is the east and Juliet is the sun' -- Romeo
\3 `Deny thy father and refuse thy name' -- Juliet
\3 `What love can do, that dares love attempt' -- Romeo
\3 `My bounty is as boundless as the sea, my love as deep' -- Juliet
\3 `Parting is such sweet sorrow' -- Juliet
\3 `Then plainly know my dear love's heart is set on the fair daughter of rich Capulet' -- Romeo (to Friar Laurence)
\3 `For this alliance may so happy prove, to turn your household's rancorr to pure love' -- Friar Laurence
\3 `These violent delights have violent ends' -- Friar Laurence
\1 Act 3
\2 Scene 1
\3 Mercutio and Benvolio are in the town square and they are worried that the Capulets will find them and start a fight
\3 Tybalt and the other Capulets enter and find the pair
\3 Mercutio teases Tybalt
\3 Romeo enters and refuses to fight as he is now related to Tybalt
\3 Mercutio is angry that Romeo is not defending himself and challenges Tybalt to fight
\3 Romeo steps between them to try to stop the fight and Tybalt strikes Mercutio, wounding him
\3 As Mercutio lies dying he curses both families
\3 Tybalt enters again and Romeo (in anger) kills him. Romeo runs away
\3 The Prince announces Romeo's banishment
\2 Scene 2
\3 Juliet is waiting in her chambers
\3 The nurse arrives and announces Tybalt's death and Romeo's banishment
\3 Juliet defends Romeo as she believes that he was defending himself and that Tybalt would have killed him
\2 Scene 3
\3 Romeo is hiding in Friar Laurence's cell
\3 He learns of his banishment, which Romeo describes as being worse than death
\3 The Friar tells Romeo to go be with Juliet, but in the morning to leave for Mantua until he can convince the Prince to pardon him
\2 Scene 5
\3 Romeo and Juliet spend the night together
\3 It is daybreak and Romeo needs to leave
\3 Juliet hopes that fate will send him back to her soon
\3 Lady Capulet tells Juliet that they have arranged her marriage to Paris next Thursday
\3 She refuses and Capulet becomes angry and states that she will either marry or he will disown her
\2 KEY QUOTES
\3 `Romeo\dots thou art a villain' -- Tybalt
\3 `A plague on both your houses' -- Mercutio
\3 `O sweet Juliet, Thy beauty has made me effeminate' -- Romeo
\3 `I am fortune's fool' -- Romeo
\3 `There is no world without Verona walls, but purgatory, torture, hell itself' -- Romeo
\3 `An you be mine, I'll give you to my friend; An you be not, hang, beg, starve, die in the streets' -- Capulet
\1 Act 4
\2 Scene 1
\3 Paris visits Friar Laurence asking to arrange his marriage to Juliet
\3 Juliet enters and is cold towards Paris. They speak of love, but Juliet's love is towards Romeo, not Paris
\3 Juliet reveals that she is desperate and threatens to kill herself if he can't help her
\3 The Friar gives Juliet a potion and explains that she will appear dead for 42 hours. When she wakes, Romeo will come and rescue her so that they can be together
\2 Scene 3
\3 Juliet and Nurse are getting her wedding outfit ready
\3 She convinces her mother and Nurse to leave her alone for the rest of the night
\3 She becomes afraid, worrying that the potion won't work or if she wakes, will she be frightened of the spirits in the tomb
\3 She drinks the potion
\2 Scene 5
\3 The nurse goes to wake Juliet, only to find her dead on her bed
\3 Capulet, Lady Capulet, servants and Paris all despair on Juliet's death
\3 Friar reassures them that she is now in heaven
\3 Music is beginning to play to signify hope
\2 KEY QUOTES
\3 `How if, when I am laid into the tomb, I wake before the time that Romeo come to redeem me?' -- Juliet
\3 `Death is my son-in-law, Death is my heir; My daughter he has wedded' -- Capulet
\3 `Most detestable Death, by thee beguiled.' -- Paris
\3 `All things that we ordain'd festival, Turn from their office to black funeral.' -- Capulet
\1 Act 5
\2 Scene 1
\3 Romeo is Mantua and talks of a dream when he seemed dead and Juliet kissed him
\3 Balthasar arrives with the news that Juliet is dead
\3 Romeo knows of an apothecary who is poor and asks him for fast-acting poison
\2 Scene 2
\3 Friar John returns to Verona
\3 He reveals to Friar Laurence that he was unable to deliver the letter to Romeo
\3 Friar Laurence realises that he has to hide in the tomb and wait for Juliet to awaken, until Romeo can come
\2 Scene 3
\3 Paris arrives at Juliet's tomb
\3 Romeo and Balthasar enter. Romeo threatens Balthasar and pleads to be left alone
\3 Paris confronts Romeo, they fight and Romeo kills him
\3 Romeo sees Juliet in the tomb, he bids her farewell and takes the poison, kissing her one last time before he dies
\3 The Friar enters the tomb to find the two bodies inside (Romeo and Paris), Juliet wakes up, and when she sees Romeo refuses to leave
\3 She takes Romeo's dagger and stabs herself, dying beside him
\3 The Prince arrives and blames Capulet and Montague
\3 Capulet and Montague unite in their grief
\2 KEY QUOTES
\3 `Thus I enforce thy rotten jaws to open, And in depite I'll cram thee with more food' -- Romeo
\3 `Beauty's ensign yet, Is crimson in thy lips and thy cheeks, And death's pale flag is not advanced there' -- Romeo
\3 `O happy dagger! This is thy sheath; there rust, and let me die' -- Juliet
\3 `Capulet, Montague. See what a scourge is laid upon your hate, that heaven find means to kill your joys with love.' -- Prince
\3 `For there never was a story of more woe, than this of Juliet and her Romeo' -- Prince
\end{outline}
\end{document}
